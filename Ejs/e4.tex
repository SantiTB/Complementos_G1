\item Sea $A$ un conjunto de números reales no vacío y acotado inferiormente. Definamos $-A:=\{-x/x\in A\}$. Probar que \[\text{inf}A=\text{-sup}(-A),\quad\text{sup}A=\text{-inf}(-A)\]
    $inf(A)$ es cota inferior de $A$, es decir $inf(A)\leq a$ para todo $a\in A$. Se sigue que $-inf(A)\geq-a$ para todo $-a\in-A$. Por ende, $-inf(A)$ es cota superior de $-A$. Sea $x\in\R$ tal que $x<-inf(A)$, entonces $-x>inf(A)$. Como $-x$ no es cota inferior de $A$, existe un $a_0\in A$ tal que $-x>a_0>inf(A)$, entonces $x<-a_0$ y $-a_0\in -A$. Se sigue que $x$ no es cota superior de $-A$. Debido a que $x$ es un real arbitrario tal que $x<-inf(A)$(cota superior de $-A$) y $x\notin (-A)^u$ se concluye que $-inf(A)=sup(-A)$.\\
    La otra igualdad se puede probar de una manera muy similar. 